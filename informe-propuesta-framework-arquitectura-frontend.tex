\documentclass[12pt,a4paper]{article}
\usepackage[spanish]{babel}
\usepackage[utf8]{inputenc}
\usepackage[T1]{fontenc}
\usepackage{geometry}
\usepackage{fancyhdr}
\usepackage{graphicx}
\usepackage{xcolor}
\usepackage{hyperref}
\usepackage{listings}
\usepackage{tabularx}
\usepackage{booktabs}

% Configuración de página
\geometry{left=2.5cm, right=2.5cm, top=2.5cm, bottom=2.5cm}
\pagestyle{fancy}
\fancyhf{}
\fancyhead[L]{\textbf{MIMP - Plataforma Visible}}
\fancyhead[R]{\textbf{Propuesta Framework y Arquitectura}}
\fancyfoot[C]{\thepage}

% Colores corporativos MIMP
\definecolor{mimpPrimary}{RGB}{139, 92, 246}
\definecolor{mimpSecondary}{RGB}{236, 72, 153}
\definecolor{mimpAccent}{RGB}{16, 185, 129}

% Configuración de enlaces
\hypersetup{
    colorlinks=true,
    linkcolor=mimpPrimary,
    urlcolor=mimpSecondary,
    citecolor=mimpAccent
}

% Configuración de listings
\lstdefinestyle{codigo}{
    backgroundcolor=\color{gray!10},
    commentstyle=\color{green!60!black},
    keywordstyle=\color{mimpPrimary},
    numberstyle=\tiny\color{gray},
    stringstyle=\color{mimpSecondary},
    basicstyle=\footnotesize\ttfamily,
    breakatwhitespace=false,
    breaklines=true,
    keepspaces=true,
    numbers=left,
    numbersep=5pt,
    showspaces=false,
    showstringspaces=false,
    showtabs=false,
    tabsize=2,
    frame=single,
    rulecolor=\color{mimpPrimary}
}

\lstset{style=codigo}

\title{
    \vspace{-2cm}
    \textbf{\Large INFORME DE PROPUESTA} \\
    \textbf{\LARGE Framework y Arquitectura Frontend} \\
    \textbf{\Large Plataforma Visible MIMP} \\
    \vspace{0.5cm}
    \textcolor{mimpPrimary}{\rule{\textwidth}{2pt}}
}

\author{
    \textbf{Ministerio de la Mujer y Poblaciones Vulnerables} \\
    \textbf{Dirección General de Políticas y Estrategias} \\
    \textit{Equipo de Desarrollo Tecnológico}
}

\date{\today}

\begin{document}

\maketitle

\section*{Resumen Ejecutivo}

El presente informe propone la implementación de una arquitectura frontend modulith para la Plataforma Visible del Ministerio de la Mujer y Poblaciones Vulnerables (MIMP), diseñada para soportar seis módulos temáticos especializados con capacidad de escalamiento futuro hacia microservicios. La solución combina \textbf{Next.js 14+}, \textbf{TypeScript} y \textbf{Clean Architecture} en un enfoque modular que maximiza la reutilización de código mientras mantiene separación lógica entre dominios de negocio.

\tableofcontents
\newpage

\section{Introducción y Contexto}

\subsection{Problemática Actual}

El Ministerio de la Mujer y Poblaciones Vulnerables requiere una plataforma digital unificada que integre información de seis áreas temáticas críticas: \textit{Niñez y Adolescencia}, \textit{Violencia de Género}, \textit{Discapacidad}, \textit{Familia}, \textit{Acoso Político} y \textit{Adulto Mayor}. Cada módulo presenta necesidades específicas de visualización de datos, workflows de moderación, y interfaces de usuario especializadas, pero debe mantener coherencia visual y arquitectónica dentro del ecosistema MIMP.

\subsection{Objetivos Estratégicos}

\begin{itemize}
    \item \textbf{Unificación}: Portal único que centralice información dispersa en múltiples sistemas
    \item \textbf{Escalabilidad}: Arquitectura preparada para crecimiento orgánico y adición de nuevos módulos
    \item \textbf{Reutilización}: Maximizar aprovechamiento de componentes entre módulos temáticos
    \item \textbf{Mantenibilidad}: Estructura de código que facilite evolución y debugging
    \item \textbf{Performance}: Optimización para usuarios gubernamentales y ciudadanía
    \item \textbf{Accessibilidad}: Cumplimiento WCAG 2.1 AA para inclusión digital
\end{itemize}

\section{Propuesta de Arquitectura Frontend}

\subsection{Arquitectura Modulith Clean}

La propuesta implementa una \textbf{arquitectura modulith} que combina ventajas del desarrollo monolítico (simplicidad inicial, deployment unificado) con separación lógica propia de microservicios (aislamiento de dominios, capacidad de extracción futura).

\begin{lstlisting}[language=bash, caption=Estructura Modular Propuesta]
sistema-visible/
├── src/
│   ├── shared/              # Componentes reutilizables
│   ├── core/               # Lógica central (auth, api, store)
│   ├── modules/            # Módulos independientes
│   │   ├── public/         # Portal público
│   │   ├── backoffice/     # Administración
│   │   ├── content/        # 6 submódulos de contenido
│   │   └── moderation/     # Moderación y analytics
│   └── app/               # Next.js App Router
\end{lstlisting}

Cada módulo implementa \textbf{Clean Architecture} con cuatro capas bien diferenciadas:

\begin{table}[h!]
\centering
\begin{tabularx}{\textwidth}{|l|X|}
\hline
\textbf{Capa} & \textbf{Responsabilidad} \\
\hline
\textit{Domain} & Entidades, reglas de negocio, interfaces de repositorios \\
\hline
\textit{Application} & Casos de uso, servicios de aplicación, orquestación \\
\hline
\textit{Infrastructure} & Adaptadores API, servicios externos, persistencia \\
\hline
\textit{Presentation} & Componentes React, páginas, hooks personalizados \\
\hline
\end{tabularx}
\caption{Capas de Clean Architecture por Módulo}
\end{table}

\subsection{Stack Tecnológico Propuesto}

La selección tecnológica prioriza estabilidad, performance y capacidad de evolución:

\begin{table}[h!]
\centering
\begin{tabularx}{\textwidth}{|l|l|X|}
\hline
\textbf{Categoría} & \textbf{Tecnología} & \textbf{Justificación} \\
\hline
\textit{Framework} & Next.js 14+ & SSR/SSG, App Router, Image Optimization, Performance \\
\hline
\textit{Lenguaje} & TypeScript & Type Safety, Mejor DX, Menos errores en producción \\
\hline
\textit{Estilos} & Tailwind CSS & Utility-first, Consistencia, Performance \\
\hline
\textit{Estado} & Zustand & Simplicidad, Performance, TypeScript-first \\
\hline
\textit{Data Fetching} & SWR/TanStack Query & Cache inteligente, Optimistic updates \\
\hline
\textit{Validación} & Zod & Schema validation, Type inference \\
\hline
\textit{UI Components} & Radix UI & Accesibilidad, Primitivas headless \\
\hline
\end{tabularx}
\caption{Stack Tecnológico Frontend}
\end{table}

\subsection{Justificación Next.js como Framework Principal}

\textbf{Next.js 14+} se selecciona como framework principal basado en los siguientes criterios técnicos y estratégicos:

\paragraph{Ventajas Técnicas:}
\begin{itemize}
    \item \textbf{App Router}: Arquitectura de routing moderna con layouts anidados y streaming
    \item \textbf{Server Components}: Reducción significativa del bundle JavaScript
    \item \textbf{Image Optimization}: Compresión y lazy loading automático de imágenes
    \item \textbf{TypeScript Integration}: Soporte nativo con type checking automático
    \item \textbf{Performance}: Core Web Vitals optimizados out-of-the-box
\end{itemize}

\paragraph{Ventajas Estratégicas:}
\begin{itemize}
    \item \textbf{Ecosistema}: Amplio ecosystem con soluciones probadas
    \item \textbf{Comunidad}: Documentación extensa y community support
    \item \textbf{Vercel Integration}: Deployment optimizado para aplicaciones gubernamentales
    \item \textbf{SEO}: Server-side rendering nativo para contenido público
\end{itemize}

\section{Arquitectura de Módulos Especializados}

\subsection{Módulos del Sistema}

La plataforma se estructura en cuatro módulos principales que encapsulan funcionalidades específicas:

\paragraph{1. Módulo Portal Público}
\begin{itemize}
    \item \textit{Objetivo}: Interfaz ciudadana para consulta de información pública
    \item \textit{Componentes}: Hero section, navegación por módulos, búsqueda global, storytelling
    \item \textit{Tecnologías específicas}: Server-side rendering, SEO optimization
\end{itemize}

\paragraph{2. Módulo Backoffice Administrativo}
\begin{itemize}
    \item \textit{Objetivo}: Panel de control para funcionarios gubernamentales
    \item \textit{Componentes}: Dashboards por rol, configuración de visibilidad, gestión de usuarios
    \item \textit{Roles}: Administrador, Editor General, Editor Observatorio
\end{itemize}

\paragraph{3. Módulos de Contenido (6 submódulos)}
\begin{itemize}
    \item \textit{Estadísticas}: Integración PowerBI, visualizaciones interactivas
    \item \textit{Servicios}: Directorio institucional con geolocalización
    \item \textit{Buenas Prácticas}: Participación ciudadana con moderación
    \item \textit{Opiniones de Expertos}: Contenido académico curado
    \item \textit{Publicaciones}: Biblioteca digital con gestión de PDFs
    \item \textit{Derechos}: Marco legal estructurado por beneficiarios
\end{itemize}

\paragraph{4. Módulo de Moderación y Analytics}
\begin{itemize}
    \item \textit{Objetivo}: Workflows de aprobación de contenido y métricas de uso
    \item \textit{Componentes}: Cola de moderación, analytics dashboard, sistema de logs
\end{itemize}

\subsection{Componentes Compartidos (Shared)}

El sistema de design system unificado garantiza coherencia visual y funcional:

\begin{lstlisting}[language=javascript, caption=Arquitectura de Componentes Compartidos]
/src/shared/
├── components/
│   ├── ui/           # Button, Input, Modal, Table
│   ├── layout/       # Header, Footer, Sidebar
│   ├── forms/        # FormField, Validation
│   └── feedback/     # Loading, Toast, ErrorBoundary
├── hooks/            # useAuth, useApi, useDebounce
├── types/            # Interfaces globales TypeScript
└── utils/            # Formatters, validators, constants
\end{lstlisting}

\section{Gestión de Estado y Data Flow}

\subsection{Arquitectura de Estado con Zustand}

La gestión de estado se implementa mediante \textbf{Zustand} con separación por dominio:

\begin{lstlisting}[language=typescript, caption=Stores Especializados]
interface AppState {
  // Estado de autenticación
  auth: {
    user: User | null;
    permissions: Permission[];
    login: (credentials: LoginData) => Promise<void>;
    logout: () => void;
  };
  
  // Estado de módulos
  modules: {
    visibilityConfig: ModuleVisibility[];
    updateVisibility: (config: ModuleVisibility) => void;
  };
  
  // Estado de contenido
  content: {
    moderationQueue: Content[];
    approveContent: (id: string) => Promise<void>;
  };
  
  // Estado de UI
  ui: {
    activeModal: string | null;
    notifications: Notification[];
    sidebarCollapsed: boolean;
  };
}
\end{lstlisting}

\subsection{Data Fetching Strategy}

La comunicación con APIs se optimiza mediante \textbf{SWR} con estrategias diferenciadas:

\begin{table}[h!]
\centering
\begin{tabularx}{\textwidth}{|l|l|X|}
\hline
\textbf{Tipo de Dato} & \textbf{Estrategia} & \textbf{Implementación} \\
\hline
\textit{Configuración} & Cache largo & SWR con revalidación cada 5 minutos \\
\hline
\textit{Contenido público} & Cache medio & SWR con stale-while-revalidate \\
\hline
\textit{Moderación} & Tiempo real & SWR con polling cada 30 segundos \\
\hline
\textit{Analytics} & Cache inteligente & SWR con invalidación por eventos \\
\hline
\end{tabularx}
\caption{Estrategias de Data Fetching}
\end{table}

\section{Sistema de Permisos y Seguridad}

\subsection{Arquitectura de Autenticación}

El sistema implementa autenticación centralizada con tres roles diferenciados:

\begin{lstlisting}[language=typescript, caption=Sistema de Roles y Permisos]
enum UserRole {
  ADMIN = 'administrator',           // Acceso completo
  EDITOR_GENERAL = 'editor_general', // Moderación global
  EDITOR_OBSERVATORIO = 'editor_observatorio' // Módulo específico
}

interface Permission {
  resource: string;    // 'content', 'users', 'config'
  action: string;      // 'create', 'read', 'update', 'delete'
  scope?: string;      // 'own', 'module', 'global'
}
\end{lstlisting}

\subsection{Seguridad Frontend}

\begin{itemize}
    \item \textbf{XSS Protection}: Sanitización automática de contenido HTML
    \item \textbf{CSRF Protection}: Tokens en formularios críticos
    \item \textbf{Content Security Policy}: Headers restrictivos para recursos externos
    \item \textbf{Input Validation}: Validación client-side y server-side con Zod
    \item \textbf{Audit Logging}: Registro de acciones sensibles para compliance
\end{itemize}

\section{Performance y Optimización}

\subsection{Estrategias de Performance}

\paragraph{Code Splitting por Módulo}
\begin{lstlisting}[language=javascript, caption=Dynamic Imports por Módulo]
// Lazy loading de módulos completos
const PublicModule = dynamic(() => import('@/modules/public'), {
  loading: () => <ModuleLoader />,
  ssr: true
});

const BackofficeModule = dynamic(() => import('@/modules/backoffice'), {
  loading: () => <AdminLoader />,
  ssr: false // Admin panel no requiere SSR
});
\end{lstlisting}

\paragraph{Optimización de Imágenes}
\begin{itemize}
    \item \textbf{Next.js Image}: Compresión automática y responsive images
    \item \textbf{Lazy Loading}: Carga diferida de imágenes below-the-fold
    \item \textbf{WebP/AVIF}: Formatos modernos con fallbacks automáticos
    \item \textbf{CDN Integration}: Distribución optimizada de assets estáticos
\end{itemize}

\subsection{Métricas de Performance Target}

\begin{table}[h!]
\centering
\begin{tabularx}{\textwidth}{|l|c|X|}
\hline
\textbf{Métrica} & \textbf{Target} & \textbf{Estrategia de Optimización} \\
\hline
\textit{First Contentful Paint} & < 1.5s & Server-side rendering, CDN \\
\hline
\textit{Largest Contentful Paint} & < 2.5s & Image optimization, code splitting \\
\hline
\textit{Cumulative Layout Shift} & < 0.1 & Skeleton screens, fixed dimensions \\
\hline
\textit{Time to Interactive} & < 3.5s & Progressive enhancement, lazy loading \\
\hline
\end{tabularx}
\caption{Core Web Vitals Targets}
\end{table}

\section{Testing y Quality Assurance}

\subsection{Estrategia de Testing Multi-Capa}

\paragraph{1. Unit Testing}
\begin{itemize}
    \item \textbf{Framework}: Jest + Testing Library
    \item \textbf{Cobertura}: 80\% mínimo en lógica de negocio
    \item \textbf{Enfoque}: Hooks personalizados, utilidades, validaciones
\end{itemize}

\paragraph{2. Integration Testing}
\begin{itemize}
    \item \textbf{Framework}: Cypress Component Testing
    \item \textbf{Cobertura}: Interacciones entre componentes complejos
    \item \textbf{Enfoque}: Formularios, workflows de moderación
\end{itemize}

\paragraph{3. End-to-End Testing}
\begin{itemize}
    \item \textbf{Framework}: Playwright
    \item \textbf{Cobertura}: User journeys críticos por rol
    \item \textbf{Enfoque}: Login, publicación de contenido, moderación
\end{itemize}

\section{Deployment y DevOps}

\subsection{Pipeline de CI/CD}

\begin{lstlisting}[language=yaml, caption=GitHub Actions Workflow]
name: Deploy Plataforma Visible
on:
  push:
    branches: [main, staging]

jobs:
  test:
    runs-on: ubuntu-latest
    steps:
      - uses: actions/checkout@v4
      - run: npm ci
      - run: npm run type-check
      - run: npm run test:unit
      - run: npm run test:e2e

  build:
    needs: test
    runs-on: ubuntu-latest
    steps:
      - run: npm run build
      - run: npm run export
      
  deploy:
    needs: build
    runs-on: ubuntu-latest
    steps:
      - name: Deploy to Staging
        if: github.ref == 'refs/heads/staging'
      - name: Deploy to Production
        if: github.ref == 'refs/heads/main'
\end{lstlisting}

\subsection{Ambientes de Deploy}

\begin{table}[h!]
\centering
\begin{tabularx}{\textwidth}{|l|l|X|}
\hline
\textbf{Ambiente} & \textbf{Propósito} & \textbf{Configuración} \\
\hline
\textit{Development} & Desarrollo local & Hot reload, source maps, debugging \\
\hline
\textit{Staging} & Testing pre-producción & Build optimizado, datos sintéticos \\
\hline
\textit{Production} & Usuario final & SSG, CDN, monitoring, analytics \\
\hline
\end{tabularx}
\caption{Ambientes de Deployment}
\end{table}

\section{Roadmap de Implementación}

\subsection{Fases de Desarrollo}

\paragraph{Fase 1: Infraestructura Base (4 semanas)}
\begin{itemize}
    \item Setup inicial Next.js + TypeScript
    \item Sistema de autenticación y roles
    \item Design system y componentes base
    \item Configuración de CI/CD
\end{itemize}

\paragraph{Fase 2: Portal Público (6 semanas)}
\begin{itemize}
    \item Hero section y navegación
    \item Sistema de búsqueda global
    \item Storytelling por módulos
    \item Optimización SEO
\end{itemize}

\paragraph{Fase 3: Backoffice Administrativo (8 semanas)}
\begin{itemize}
    \item Dashboards diferenciados por rol
    \item Configuración de visibilidad
    \item Sistema de moderación
    \item Analytics dashboard
\end{itemize}

\paragraph{Fase 4: Módulos de Contenido (12 semanas)}
\begin{itemize}
    \item Implementación de 6 submódulos especializados
    \item Integración con PowerBI
    \item Sistema de archivos y validaciones
    \item Workflows de aprobación
\end{itemize}

\paragraph{Fase 5: Optimización y Testing (4 semanas)}
\begin{itemize}
    \item Performance optimization
    \item Accessibility compliance
    \item Security hardening
    \item Load testing
\end{itemize}

\section{Conclusiones y Recomendaciones}

\subsection{Beneficios de la Arquitectura Propuesta}

\begin{enumerate}
    \item \textbf{Escalabilidad Técnica}: La arquitectura modulith permite crecimiento orgánico desde un monolito hacia microservicios cuando los requerimientos de escala lo justifiquen.
    
    \item \textbf{Reutilización de Código}: El sistema de componentes compartidos maximiza el aprovechamiento entre módulos, reduciendo tiempo de desarrollo y mejorando consistency.
    
    \item \textbf{Separación de Responsabilidades}: Clean Architecture facilita testing, mantenimiento y onboarding de nuevos desarrolladores.
    
    \item \textbf{Performance Optimizada}: Next.js proporciona optimizaciones automáticas que impactan directamente en métricas gubernamentales de servicio digital.
    
    \item \textbf{Type Safety}: TypeScript reduce errores en producción y mejora la experiencia de desarrollo en equipos grandes.
\end{enumerate}

\subsection{Recomendaciones Estratégicas}

\paragraph{1. Adopción Gradual}
Implementar la migración por módulos, comenzando con funcionalidades de menor riesgo y evolucionando hacia workflows críticos.

\paragraph{2. Capacitación del Equipo}
Invertir en training específico en Next.js, TypeScript y Clean Architecture para maximizar adopción y calidad de código.

\paragraph{3. Monitoring y Observabilidad}
Implementar desde el inicio sistemas de logging, metrics y tracing que faciliten debugging y optimización continua.

\paragraph{4. Security First}
Establecer políticas de security review en cada pull request y automated security scanning en el pipeline CI/CD.

\subsection{Consideraciones de Riesgo}

\begin{itemize}
    \item \textbf{Learning Curve}: TypeScript y Clean Architecture requieren inversión inicial en capacitación
    \item \textbf{Complejidad Inicial}: La separación por capas añade overhead conceptual
    \item \textbf{Vendor Lock-in}: Next.js tiene optimizaciones específicas para Vercel deployment
\end{itemize}

\subsection{Mitigación de Riesgos}

\begin{itemize}
    \item \textbf{Documentación Exhaustiva}: Mantener documentación arquitectónica y patrones de código
    \item \textbf{Code Review Riguroso}: Establecer guidelines claros y review process
    \item \textbf{Alternative Deployment}: Mantener compatibilidad con múltiples plataformas de hosting
\end{itemize}

\section*{Aprobación del Documento}

\vspace{2cm}

\noindent
\begin{minipage}{0.45\textwidth}
\textbf{Elaborado por:}\\[0.5cm]
\rule{5cm}{0.4pt}\\
Equipo de Desarrollo Tecnológico\\
MIMP
\end{minipage}
\hfill
\begin{minipage}{0.45\textwidth}
\textbf{Aprobado por:}\\[0.5cm]
\rule{5cm}{0.4pt}\\
Director General de Políticas\\
y Estrategias - MIMP
\end{minipage}

\vspace{1cm}
\noindent
\textit{Fecha de elaboración: \today}

\end{document}